\documentclass{article}\usepackage[]{graphicx}\usepackage[]{color}
%% maxwidth is the original width if it is less than linewidth
%% otherwise use linewidth (to make sure the graphics do not exceed the margin)
\makeatletter
\def\maxwidth{ %
  \ifdim\Gin@nat@width>\linewidth
    \linewidth
  \else
    \Gin@nat@width
  \fi
}
\makeatother

\definecolor{fgcolor}{rgb}{0.345, 0.345, 0.345}
\newcommand{\hlnum}[1]{\textcolor[rgb]{0.686,0.059,0.569}{#1}}%
\newcommand{\hlstr}[1]{\textcolor[rgb]{0.192,0.494,0.8}{#1}}%
\newcommand{\hlcom}[1]{\textcolor[rgb]{0.678,0.584,0.686}{\textsf{#1}}}%
\newcommand{\hlopt}[1]{\textcolor[rgb]{0,0,0}{#1}}%
\newcommand{\hlstd}[1]{\textcolor[rgb]{0.345,0.345,0.345}{#1}}%
\newcommand{\hlkwa}[1]{\textcolor[rgb]{0.161,0.373,0.58}{\textbf{#1}}}%
\newcommand{\hlkwb}[1]{\textcolor[rgb]{0.69,0.353,0.396}{#1}}%
\newcommand{\hlkwc}[1]{\textcolor[rgb]{0.333,0.667,0.333}{#1}}%
\newcommand{\hlkwd}[1]{\textcolor[rgb]{0.737,0.353,0.396}{\textbf{#1}}}%

\usepackage{framed}
\makeatletter
\newenvironment{kframe}{%
 \def\at@end@of@kframe{}%
 \ifinner\ifhmode%
  \def\at@end@of@kframe{\end{minipage}}%
  \begin{minipage}{\columnwidth}%
 \fi\fi%
 \def\FrameCommand##1{\hskip\@totalleftmargin \hskip-\fboxsep
 \colorbox{shadecolor}{##1}\hskip-\fboxsep
     % There is no \\@totalrightmargin, so:
     \hskip-\linewidth \hskip-\@totalleftmargin \hskip\columnwidth}%
 \MakeFramed {\advance\hsize-\width
   \@totalleftmargin\z@ \linewidth\hsize
   \@setminipage}}%
 {\par\unskip\endMakeFramed%
 \at@end@of@kframe}
\makeatother

\definecolor{shadecolor}{rgb}{.97, .97, .97}
\definecolor{messagecolor}{rgb}{0, 0, 0}
\definecolor{warningcolor}{rgb}{1, 0, 1}
\definecolor{errorcolor}{rgb}{1, 0, 0}
\newenvironment{knitrout}{}{} % an empty environment to be redefined in TeX

\usepackage{alltt}

\usepackage[utf8]{inputenc}
%\usepackage[cp1251]{inputenc}
\usepackage[english]{babel}
\usepackage{indent first}
\usepackage[top=2 cm, bottom = 2 cm, left = 3 cm, right = 1.5 cm]{geometry}
\usepackage{amsmath}
\usepackage{amssymb}
\usepackage{amsthm}
\usepackage{graphicx}
\usepackage{wrapfig}
\usepackage{listings}
\usepackage{ mathrsfs }
\usepackage{subcaption}
\graphicspath{{images/}}
\renewcommand{\baselinestretch}{1.1}

\newtheorem*{theorem}{Theorem}
\IfFileExists{upquote.sty}{\usepackage{upquote}}{}
\begin{document}
%<*includetag>
\section{Assignment 4}
\subsection{Selection and building nominal features}
In our dataset we have several binary features, such as weekdays (\texttt{weekday\_is\_monday}, \texttt{weekday\_is\_tuesday} and so on) and  belonging to one of the channels (\texttt{data\_channel\_is\_lifestyle}, \texttt{data\_channel\_is\_entertainment} and so on). Therefore we built two nominal features:
\begin{itemize}
\item channel: integer values ranging between 1 and 6 ('Lifestyle', 'Entertainment', 'Business', 'Social Media', 'Tech', 'World')
\item weekday: integer values ranging between 1 and 7 
\end{itemize}

To obtain the third nominal feature we divide the feature \texttt{timedelta} into four parts : days between the article publication and the dataset acquisition.
\begin{knitrout}
\definecolor{shadecolor}{rgb}{0.969, 0.969, 0.969}\color{fgcolor}\begin{kframe}
\begin{alltt}  
\hlstd{timegroup} \hlkwb{<-} \hlkwa{cut}\hlstd{(data$timedelta, breaks = 4})
\end{alltt}
\end{kframe}
\end{knitrout} 

And we break range of values of \texttt{timedelta} into intervals of approximate equal size: $(7.28,189]$, $(189,370]$, $(370,550]$, $(550,732]$. 
Let us note, that analysis of cross classification between timedelta and channel is 

\subsection{Contingency tables over features}
Conditional cross-classification talbes between introduced nominal fetures are obtained with R-function \texttt{table} as is shown below. Results are presented in Tables \ref{tbl:cont_table_ch_timed} -- \ref{tbl:cont_table_week_timegroup}. 
Simple analysis of the aforementioned tables reveals that there aren't any conceptual associations between categories. So our analisys could go deeper and discover some hidden dependencies with techinques that were explained in lectures.



\begin{knitrout}
\definecolor{shadecolor}{rgb}{0.969, 0.969, 0.969}\color{fgcolor}\begin{kframe}
\begin{alltt}  
table(data$channel, data$timegroup)
table(data$channel, data$weekday)
\end{alltt}
\end{kframe}
\end{knitrout} 

\begin{table}[h]
\begin{center}
\small
\begin{minipage}[h]{0.45\linewidth}
\caption{Cross classification of the \texttt{channel} with \texttt{timegroup}} \label{tbl:cont_table_ch_timed}
\begin{tabular}{|c|c|c|c|c|} 
 \hline
  &(7.28,189] &(189,370]& (370,550] &(550,732] \\ \hline
  0  &      412   &    327   &    361   &    381 \\ \hline
  1  &      120   &     93   &    130   &    183 \\ \hline
  2  &      573   &    473   &    341   &    348 \\ \hline
  3  &      392   &    408   &    410   &    446 \\ \hline
  4  &       81   &    154   &    163   &    192 \\ \hline
  5  &      401   &    469   &    491   &    492 \\ \hline
  6  &      901   &    564   &    373   &    321 \\ \hline
\end{tabular}
\end{minipage}
\hfill 
\begin{minipage}[h]{0.45\linewidth}
\caption{Cross classification of the \texttt{channel} with \texttt{weekday}} \label{tbl:cont_table_ch_week}
\begin{tabular}{|c|c|c|c|c|c|c|c|} 
 \hline
     &1 & 2 & 3 & 4 & 5 & 6 & 7 \\ \hline
  0  & 224 & 256 &253 &257 &238 &112 &141 \\ \hline
  1  & 80 & 95 &  92 & 85  &69  &46  &59 \\ \hline
  2  & 317 & 332 & 311 & 287 & 241 & 95 & 152 \\ \hline
  3  & 277 & 293 & 371 & 319 & 245 & 57 & 94 \\ \hline
  4  & 88 & 116 &105& 115 & 88 & 44 & 34 \\ \hline
  5  & 316 & 372 & 358 & 344 & 244 & 125 & 94\\ \hline
  6  & 360& 379& 399& 392& 342& 136& 151 \\ \hline
\end{tabular}
\end{minipage}
\end{center}
\end{table}


\begin{table}[h!]
\begin{center}
\small
\caption{Cross classification of the \texttt{weekday} with \texttt{timegroup} }
\label{tbl:cont_table_week_timegroup}
\begin{tabular}{|c|c|c|c|c|} 
\hline
& (7.28,189] &(189,370] &(370,550] &(550,732]\\ \hline
1 &         57 &       40 &       30 &       36\\ \hline
2 &         48 &       41 &       37 &       45\\ \hline
3 &         51 &       49 &       48 &       40\\ \hline
4 &         41 &       45 &       49 &       48\\ \hline
5 &         46 &       35 &       29 &       33\\ \hline
6 &         21 &       13 &       16 &       17\\ \hline
7 &         20 &       24 &       24 &       17\\ \hline
\end{tabular}
\end{center}
\end{table}







\begin{table}[h]
\begin{center}
\small
\caption{Conditional frequency table over \texttt{channel} and \texttt{timegroup}}
\label{tbl:cont_table_ch_timed_freq}
\begin{tabular}{|c|c|c|c|c|c|} 
\hline
& (7.28,189] &(189,370] &(370,550] &(550,732] &   Sum \\ \hline
0   &       4.12 &     3.27 &     3.61 &     3.81 & 14.81 \\ \hline
1   &       1.20 &     0.93 &     1.30 &     1.83 &  5.26 \\ \hline
2   &       5.73 &     4.73 &     3.41 &     3.48 & 17.35 \\ \hline
3   &       3.92 &     4.08 &     4.10 &     4.46 & 16.56 \\ \hline
4   &       0.81 &     1.54 &     1.63 &     1.92 &  5.90 \\ \hline
5   &       4.01 &     4.69 &     4.91 &     4.92 & 18.53 \\ \hline
6   &       9.01 &     5.64 &     3.73 &     3.21 & 21.59 \\ \hline
Sum &      28.80 &    24.88 &    22.69 &    23.63 &100.00 \\ \hline
\end{tabular}
\end{center}
\end{table}

\begin{table}[h]
\small
\begin{center}
\caption{Conditional frequency table over \texttt{channel} and \texttt{weekday}} 
\label{tbl:cont_table_ch_week_freq}
\begin{tabular}{|c|c|c|c|c|c|c|c|c|} 
\hline
&    1 &     2 &     3 &     4 &     5  &    6 &     7 &   Sum \\ \hline
0   & 2.24 &  2.56 &  2.53 &  2.57 &  2.38  & 1.12 &  1.41 & 14.81 \\ \hline
1   & 0.80 &  0.95 &  0.92 &  0.85 &  0.69  & 0.46 &  0.59 &  5.26 \\ \hline
2   & 3.17 &  3.32 &  3.11 &  2.87 &  2.41  & 0.95 &  1.52 & 17.35 \\ \hline
3   & 2.77 &  2.93 &  3.71 &  3.19 &  2.45  & 0.57 &  0.94 & 16.56 \\ \hline
4   & 0.88 &  1.16 &  1.05 &  1.15 &  0.88  & 0.44 &  0.34 &  5.90 \\ \hline
5   & 3.16 &  3.72 &  3.58 &  3.44 &  2.44  & 1.25 &  0.94 & 18.53 \\ \hline
6   & 3.60 &  3.79 &  3.99 &  3.92 &  3.42  & 1.36 &  1.51 & 21.59 \\ \hline
Sum &16.62 & 18.43 & 18.89 & 17.99 & 14.67  & 6.15 &  7.25 &100.00 \\ \hline
\end{tabular}
\end{center}
\end{table}


\begin{table}[h]
\begin{center}
\small
\caption{Conditional frequency table over \texttt{weekday} with \texttt{timegroup} }
\label{tbl:cont_table_week_timegroup_freq}
\begin{tabular}{|c|c|c|c|c|c|} 
\hline
& (7.28,189] &(189,370] &(370,550] &(550,732] &   Sum \\ \hline
1   &        5.7 &      4.0 &      3.0 &      3.6 &  16.3 \\ \hline
2   &        4.8 &      4.1 &      3.7 &      4.5 &  17.1 \\ \hline
3   &        5.1 &      4.9 &      4.8 &      4.0 &  18.8 \\ \hline
4   &        4.1 &      4.5 &      4.9 &      4.8 &  18.3 \\ \hline
5   &        4.6 &      3.5 &      2.9 &      3.3 &  14.3 \\ \hline
6   &        2.1 &      1.3 &      1.6 &      1.7 &   6.7 \\ \hline
7   &        2.0 &      2.4 &      2.4 &      1.7 &   8.5 \\ \hline
Sum &       28.4 &     24.7 &     23.3 &     23.6 & 100.0 \\ \hline
\end{tabular}
\end{center}
\end{table}
 

 

As could be clearly seen, it is pretty cumbersome to make any conclusion about data which is presented without normalization. So, for the ease of interpretaion,  the same  data converted to relative frequencies by relating them to the total number of entities is presented in Tables \ref{tbl:cont_table_ch_timed_freq} -- \ref{tbl:cont_table_week_timegroup_freq}. 

We choose \texttt{channel}  as the common feature for comparison with two other features. Motivation for such choice is as follows, there is no much sense in cross-classification subgroups of time passed from the article publishing till dataset acquisition   and the day of the week it was published. This point of veiw is also supported by the Table \ref{tbl:cont_table_week_timegroup_freq}, wich reveals no particular irregularities of the data, except the number of publication in a particular day of week.


\clearpage
 

Quetelet relative index tables over our nominal fetures we obtain as a result of the following function:
\begin{knitrout}
\definecolor{shadecolor}{rgb}{0.969, 0.969, 0.969}\color{fgcolor}\begin{kframe}
\begin{alltt}  
getQueteletIndex \hlkwb{<-} \hlkwa{function}(v1, v2) \{
  size \hlkwb{<-} length(v1)
  cont.table  \hlkwb{<-} table(v1, v2)
  row.sums  \hlkwb{<-} rowSums(cont.table)
  col.sums  \hlkwb{<-} colSums(cont.table)
  norm.cont.table  \hlkwb{<-}  cont.table / size
  norm.row.sums  \hlkwb{<-} row.sums / size
  norm.col.sums  \hlkwb{<-} col.sums / size
  list(Quetelet = norm.cont.table / (norm.row.sums %*% t(norm.col.sums)) - 1,
       PearsonIndexMatrix = (-norm.row.sums %*% t(norm.col.sums) + norm.cont.table) /
          sqrt(norm.row.sums %*% t(norm.col.sums)))
\}
\end{alltt}
\end{kframe}
\end{knitrout}

The results (in percent) are  presented in Tables \ref{tbl:quet_table_ch_timed}, \ref{tbl:quet_table_ch_week}. 
\begin{table}[h]
\footnotesize
\begin{center}
\begin{minipage}[h]{0.4\linewidth}
\caption{Quetelet relative index table over \texttt{channel} and \texttt{timegroup}} \label{tbl:quet_table_ch_timed}
\begin{tabular}{|c|c|c|c|c|} 
 \hline
  &(7.28,189] &(189,370]& (370,550] &(550,732] \\ \hline
  0&      -3.41  &  -11.26&      7.43&      8.87 \\ \hline
  1 &    -20.79  &  -28.94&      8.92&\textbf{47.23} \\ \hline
  2  &14.67 &\textbf{9.57}&    -13.38&    -15.12 \\ \hline
  3&     -17.81 &    -0.97&      9.12&     13.98 \\ \hline
  4 &    -52.33 & 4.91&\textbf{21.76}&     37.72 \\ \hline
  5  &   -24.86&      1.73&     16.78&     12.36 \\ \hline
  6&\textbf{44.90}&   5.00&    -23.86&    -37.08 \\ \hline
\end{tabular}
\end{minipage}
\hfill 
\begin{minipage}[h]{0.5\linewidth}
\caption{Quetelet relative index table over \texttt{channel} and \texttt{weekday}} \label{tbl:quet_table_ch_week}
\begin{tabular}{|c|c|c|c|c|c|c|c|} 
 \hline
     &1 & 2 & 3 & 4 & 5 & 6 & 7 \\ \hline
  0  &-9.00&  -6.21&  -9.57&  -3.54 &  9.54&  22.97&  31.32 \\ \hline
  1 & -8.49&  -2.00&  -7.41& -10.17 &-10.58&\textbf{42.20}&\textbf{54.71}\\ \hline
  2&\textbf{9.93}&   3.83&  -5.11&  -8.05 & -5.31& -10.97&  20.84 \\ \hline
  3&   0.64&  -4.00&\textbf{18.60}&   7.08 &  0.85& -44.03& -21.71 \\ \hline
  4& -10.26&   6.68&  -5.79&\textbf{8.35}&  1.67&  21.26& -20.51 \\ \hline
  5&   2.61&\textbf{8.93}&   2.28&   3.19 &-10.24&   9.69& -30.03 \\ \hline
  6&   0.33&  -4.75&  -2.17&   0.93 &\textbf{7.98}&   2.43&  -3.53 \\ \hline
\end{tabular}
\end{minipage}
\end{center}
\end{table}

As we can see from the Table \ref{tbl:quet_table_ch_timed}, \texttt{timegroup} is dependent with \texttt{channel} in some values. For example, we observe rather big Quetelet relative index between \textsf{Entartanment channel} and \textsf{4th time-group}. \footnote{This could be explained with the hypothesis of some major entertainment event, which happened quite a long time before the dataset acquisition was occured. }
In addition, we can't reject a dependence between \textsf{World channel} and \textsf{1st time-group}. It can be caused by not random sampling or by some extra-ordinary events with great response in the world.

Table \ref{tbl:quet_table_ch_week} provides us less surprising and slightly more predictable results: all channels are almost independent with weekdays except of \textsf{Entartanment channel} and \textsf{Weekend} pair. This observation is easy to interpret: users visit Mashable at the weekend -- period when they have more free time (compared to workdays) to amuse themselves. 


\subsection{$\chi^2$--summary Quetelet index}


\begin{table}[ht]
\centering
\begin{tabular}{|c|c|c|c|c|c|}
  \hline
 &(7.28,189] & (189,370] & (370,550] & (550,732] & Sum \\ 
  \hline
0 & -0.007 (0.00005) & -0.02161 (0.0004) & 0.01362 (0.0002) & 0.01659 (0.0003) & (0.00098) \\  \hline
1 & -0.02558 (0.0006) & -0.03310 (0.001) & 0.00975 (0.0001) & 0.05266 (0.003)  & (0.0046)  \\ \hline
2 & 0.03280 (0.001) & 0.01989 (0.0004) & -0.02655 (0.0007) & -0.03061 (0.0009) & (0.003)   \\ \hline
3 & -0.03889 (0.0015) & -0.00198 (0) & 0.01767 (0.0003) & 0.02765 (0.0008)     & (0.0026)  \\ \hline
4 & -0.06821 (0.0047) & 0.00595  (0.00004) & 0.02518 (0.0006) & 0.04453 (0.002)& (0.0073) \\ \hline
5 & -0.05743 (0.0033) & 0.00371 (0.00001) & 0.03441 (0.0012) & 0.02587 (0.0006)& (0.005)  \\ \hline
6 & 0.11197 (0.013) & 0.01158 (0.00013) & -0.05281 (0.00279) & -0.08375 (0.007)& (0.0225)  \\ \hline
Sum & (0.0237)      & (0.002)			& (0.006)			 & (0.0144)		   & (0.04624) \\ \hline
\end{tabular}
\caption{$\chi^2$--summary Quetelet index over \texttt{channel} and \texttt{timegroup}} \label{tbl:quet_table_ch_week}
\end{table}

\begin{table}[ht]
\centering
\footnotesize
\begin{tabular}{|c|c|c|c|c|c|c|c|c|}
  \hline
 & 1 & 2 & 3 & 4 & 5 & 6 & 7 & Sum \\ 
  \hline
0 & -0.014  (0) & -0.01  (0) & -0.016  (0) & -0.006  (0) & 0.014  (0) & 0.022  (0) & 0.032  (0.001) & (0.002) \\  \hline
1 & -0.008  (0) & -0.002  (0) & -0.007  (0) & -0.01  (0) & -0.009  (0) & 0.024  (0.001) & 0.034  (0.001) & (0.002) \\  \hline
2 & 0.017  (0) & 0.007  (0) & -0.009  (0) & -0.014  (0) & -0.008  (0) & -0.011  (0) & 0.023  (0.001) & (0.001) \\  \hline
3 & 0.001  (0) & -0.007  (0) & 0.033  (0.001) & 0.012  (0) & 0.001  (0) & -0.044  (0.002) & -0.024  (0.001) & (0.004) \\  \hline
4 & -0.01  (0) & 0.007  (0) & -0.006  (0) & 0.009  (0) & 0.002  (0) & 0.013  (0) & -0.013  (0) & (0.001) \\  \hline
5 & 0.005  (0) & 0.017  (0) & 0.004  (0) & 0.006  (0) & -0.017  (0) & 0.01  (0) & -0.035  (0.001) & (0.002) \\  \hline
6 & 0.001  (0) & -0.009  (0) & -0.004  (0) & 0.002  (0) & 0.014  (0) & 0.003  (0) & -0.004  (0) & (0) \\  \hline
Sum & (0.001) & (0.001) & (0.002) & (0.001) & (0.001) & (0.003) & (0.005) &  (0.0124)\\ 
   \hline
\end{tabular}
\caption{$\chi^2$--summary Quetelet index over \texttt{channel} and \texttt{weekday}} \label{tbl:quet_table_ch_week}
\end{table}

\subsection{Sufficient sample size for significant result}

Supposing the probabilities $p_{i+}$, $p_{+j}$, $p_{ij}$ are constant and sample size $n$ is varying, we can get $\chi^2$-statistics from
\begin{gather*}
nX^2 = \sum\limits_{k=1}^K \sum\limits_{l=1}^L \frac{(p_{kl} - p_{k+}p_{+l})^2}{p_{k+}p_{+l}} \overset{n \to \infty}{\Rightarrow} \chi^2((K-1)(L-1))
\end{gather*} 

We know $X^2$ and $L$ for pairs \texttt{channel}--\texttt{timegroup} and \texttt{channel}--\texttt{weekday}, so, we can get sufficient $K$ for significant results.

%</includetag>
\end{document}
